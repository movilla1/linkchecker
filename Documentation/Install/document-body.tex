\section{Pre-Requisites / server requirements}
This system was developed using Laravel as the base framework, version 5.5.3, and it requires the following software to be available on the server to work:

\begin{itemize}
	\item PHP newer or equal to 7.0.0
	\item OpenSSL PHP Extension
	\item PDO PHP Extension
	\item Mbstring PHP Extension
	\item Tokenizer PHP Extension
	\item XML PHP Extension
\end{itemize}

Once all those software requirements are complete, you need to 

\begin{enumerate}
	\item create a database for the app in the server
	\item upload the source into the server
	\item open a shell console to the app
	\item cd into the app uploaded folder (eg.: cd /home/linkchecker/public\_html)
	\item run: \verb|composer install|
	\item set the web server to use app/public as the document root 
	\item edit the .env file and set the details for the app to run, specially the database details.
	\item on the app folder you need to migrate the database in order to create the needed tables, to do so, we offer two alternatives:
	\begin{itemize}
		\item \verb|php artisan migrate| 
		\item \verb|php artisan migrate --seed|\footnote{This option allows the data population with sample items for all tables}
	\end{itemize}
	\item after this, your server should show the application properly and you can begin to use it.
\end{enumerate}

Please run the \verb|--seed| option unless you know exactly what you are doing, as this will create the default user and administration accounts and they will allow you to start working.

Below we provide a copy of a sample .env file for your reference
\begin{verbatim}
APP_NAME="BackLink Checker"
APP_ENV=local
APP_KEY=base64:qQj+GQKurPZsv5dmGOFFHVckzjBu2Szt8A0vxR6j7MI=
APP_DEBUG=true
APP_LOG_LEVEL=debug
APP_URL=http://localhost

DB_CONNECTION=mysql
DB_HOST=127.0.0.1
DB_PORT=3306
DB_DATABASE=backlinkchecker
DB_USERNAME=zlinkcheck
DB_PASSWORD=55zl1nK3r

BROADCAST_DRIVER=log
CACHE_DRIVER=file
SESSION_DRIVER=file
QUEUE_DRIVER=sync

MAIL_DRIVER=smtp
MAIL_HOST=smtp.mailtrap.io
MAIL_PORT=2525
MAIL_USERNAME=null
MAIL_PASSWORD=null
MAIL_ENCRYPTION=null

PUSHER_APP_ID=
PUSHER_APP_KEY=
PUSHER_APP_SECRET=
\end{verbatim}

Feel free to use it for your own .env or modify it as needed.

If you need help installing laravel on your hosting, you can refer to this video on youtube \verb|https://www.youtube.com/watch?v=DsgWKuGk3yM|